\documentclass{article}
\usepackage{graphicx}
\usepackage[raggedrightboxes]{ragged2e}
\usepackage{hyperref}

\hypersetup{
    colorlinks=true,
    linkcolor=black
}

\title{DVIZ Data story report}
\author{David Hodel}
\date{January 2024}

\begin{document}

\maketitle
\newpage

\tableofcontents
\newpage

\section{Project overview}

For the HSLU fall semester 2023 module \textit{DVIZ: Data Visualisation for AI and Machine Learning} we were tasked to create a data story
using one or more datasets of our choice. The focus was on the storytelling and design (conscious use of color, annotations, etc.) of the charts.
The story was allowed to be static or interactive and had to include at least 4 charts. \newline
I chose to create a static data story about the changing energy landscape of the world which allows humanity to progress into a brighter future.
The first two charts show the massive increase in global electricity demand in the last 20 years, making the case that electricity is the key to a better future.
The third chart shows the change of electricity mix between 2000 and 2020, showing that the world is moving away from fossil fuels and towards renewable energy sources.
The fourth chart shows the change of cost vs. capacity for renewable energy sources, showing that the cost of renewable energy is decreasing while the capacity is increasing. \newline

\newpage

\section{Motivation}

I regularly watch the YouTube channel \textit{melodysheep} which is run by John D. Boswell, a filmmaker, composer, VFX artist, and editor from the pacific northwest. 
His work spans from television and music production to viral remixes and mashups,
and has drawn over a quarter billion views online. \newline
In August 2023 he released the video \textit{THE HUMAN FUTURE: A Case for Optimism} (\texttt{https://www.youtube.com/watch?v=o48X3\_XQ9to})
which is about the future of humanity and how we can use science and technology to make the world a better place.
The video is very well-made and made me think about ever-present bad news about war, poverty, climate change and other problems in the world. \newline
I wanted to explore this topic in more depth and see if there is any truth to the claim that the world is getting better, and if there is any data to back it up.
This motivated me to choose the topic as the subject for my data story.

\newpage

\section{Work Summary}

The following table shows the work done for this project.
The hours are estimated and rounded to the nearest full hour.

\begin{table}[!ht]
    \centering
    \begin{tabular}{|p{2cm}|p{1cm}|p{7.5cm}|}
    \hline
        \textbf{Date} & \textbf{Hours} & \textbf{Task description} \\ \hline
        19.11.2023 & 5 & Create general overview of story, create venv, start searching for datasets \\ \hline
        20.11.2023 & 3 & Data exploration for global life expectancy \\ \hline
        26.11.2023 & 3 & Data exploration for global child mortality rate and polio cases \\ \hline
        30.11.2023 & 4 & Data exploration for global internet users and renewable energy share \\ \hline
        04.12.2023 & 4 & Coaching session, rethink scope of project, begin exploring datasets for (renewable) energy consumption / production \\ \hline
        09.12.2023 & 3 & Exploring OWID dataset for global energy \\ \hline
        10.12.2023 & 3 & Find \& explore suitable dataset for renewable energy costs and energy capacity \\ \hline
        11.12.2023 & 4 & Further exploring OWID dataset and create draft outline for data story \\ \hline
        12.12.2023 & 4 & Set up Quarto, copy outline to Quarto notebook, began with annotating first chart \\ \hline
        17.12.2023 & 4 & Annotating and designing first and second chart \\ \hline
        18.12.2023 & 5 & Annotating third and fourth chart, further annotating first chart \\ \hline
        19.12.2023 & 3 & Finish annotating all the charts \\ \hline
        27.12.2023 & 5 & Restructure Jupyter Notebook, extract individual steps for using it in Quarto, layouting in Quarto article, start writing project report \\ \hline
        29.12.2023 & 3 & Continue writing the report \\ \hline
        01.01.2024 & 2 & Continue writing the report (exploratory work, acknowledgements) \\ \hline
        \hline
        Total & 55 &  \\ \hline
    \end{tabular}
\end{table}

\newpage

\section{Exploritary Work}
% description of the exploratory work you did (data exploration, an actual data analysis, but also
% doing a review of similar work that has been already done with citations - these don’t have to be
% peer-reviewed papers, but can also be web-apps, journal articles etc.) 
% citations for the data source and other relevant sources

At the beginning of the project I struggled to find a topic to write an interesting and meaningful data story about.
I explored far too many datasets and topics and did not focus on a single one.
This resulted in a lot of wasted time and effort as I had to research and understand many different datasets.
The data cleaning and exploration for each dataset was very time-consuming and didn't result in any meaningful insight as I did not have a clear goal in mind. \newline
In the beginning of December I had a coaching session with the lecturer, and we discussed my progress so far.
She strongly advised me to rethink the scope of my project and focus on a single topic.
This helped me a lot and I decided to focus on the topic of energy consumption and production as it's a central topic for the future of humanity. \newline
The following sections describe the exploratory work I did for the different datasets I explored.
The datasets for topic I decided to use in the final article are described in more detail.

\subsection{Life expectancy at birth}

Dataset URL: \texttt{https://data.worldbank.org/indicator/SP.DYN.LE00.IN} \newline
This dataset contains the life expectancy at birth in years for all countries in the world from 1960 to 2021.
I used the data to recognize the upward trend in life expectancy in all countries in the world and the average life expectancy.
I noted the biggest differences in both relative and absolute terms. 

\subsection{Infant mortality rate}

Dataset URL: \texttt{https://data.worldbank.org/indicator/SP.DYN.IMRT.IN} \newline
This dataset contains the infant mortality rate per 1000 live births for all countries in the world from 1960 to 2021.
The insights I gained from this dataset were similar to the life expectancy dataset.
the mortality rate is decreasing in all countries and the average mortality rate is decreasing as well.

\subsection{Estimated polio cases by world region}

Dataset URL: \texttt{https://ourworldindata.org/polio\#the-number-of-estimated-polio-cases-by-world-region} \newline
This dataset contains the estimated paralytic polio cases count for all countries from 1980 to 2020.
I used the data to show the massive decrease in polio cases in all regions (continents) of the world.

\subsection{Global internet users}

Dataset URL: \texttt{https://www.kaggle.com/datasets/ashishraut64/internet-users} \newline
This dataset contains the number of internet users and the internet penetration rate for all countries in the world from 1980 to 2020.
I used the data to show the increase in internet users and the internet penetration rate on a global scale.

\subsection{Literacy rate}

Dataset URL: \texttt{https://ourworldindata.org/literacy} \newline
This dataset contains the literacy rate of adults aged 15 or older for all countries in the world from 1475 to 2022.
I used the data to show the increase in literacy rate over time on a map using the years as the animation variable.
I also showed the inequality between sub-Saharan Africa and the rest of the world

\subsection{Energy demand}

Dataset URL: \texttt{https://www.kaggle.com/datasets/pralabhpoudel/world-energy-consumption/data} \newline

\subsection{Energy capacity}

Dataset URL:  \newline

\subsection{Renewable energy costs}

Dataset URL:  \newline

\newpage

\section{Data visualizations}
% explanation why you chose particular chart types and other visualization elements like colors, annotations, insets, styling
My data story targets a broad audience that ranges from environmentally conscious individuals and policymakers to business leaders and educators.
The charts are made to be easy to understand and show the important points right away.
This helps everyone get the big picture about strong increases of electricity demand and moving to cleaner energy sources like wind and solar power. \newline
For my data story I chose to create four charts. The following sections describe the charts and the reasons for choosing particular types and visualization elements.

\subsection{Chart 1: Global electricity demand over time}

\begin{itemize}
    \item Shows the global electricity demand from 2000 to 2020 using a stacked area chart
    \item I chose a stacked area chart because it allows me to show the evolution of the whole and the relative proportions of the different regions
    \item For the regions I considered the following: Africa, Asia, Europe, North and South America and Oceania. These regions are also used in the OWID dataset and are a good representation of the world.
    \item The colors for the regions are consciously chosen to be easily distinguishable and to have a connection with attributes of the continents:
        \begin{itemize}
            \item Africa: Brown. Represents the vast landscapes of Africa, including its deserts (like the Sahara), wildlife, and fertile lands.
            \item Asia: Red. Is a color of significance in many Asian cultures. In China, for example, it symbolizes good luck, joy, and prosperity.
            \item Europe: Blue. Because of the blue flag of the European Union
            \item North America: Green. Because of its natural landscapes, especially the forests and the Rocky Mountains
            \item South America: Orange. Vibrant cultures, warm climates and diverse ecosystemns
            \item Oceania: Teal. Represents the unique and diverse marine environment of Oceania, especially the extensive  coral reefs like the Great Barrier Reef in Australia
        \end{itemize}
    \item The colors are muted and not too bright to not distract from the actual data
    \item The chart is annotated with the most important insights:
    \begin{itemize}
        \item The total electricity demand has increased by 76\% from 2000 to 2020
        \item There were two small dips in 2008 and 2020 due to big global events
    \end{itemize}
\end{itemize}


\subsection{Chart 2: Electricity demand per region}

\begin{itemize}
    \item Shows the electricity demand for each region in 2000 and 2020 using a grouped bar chart
    \item I chose a grouped bar chart because it allows me to directly compare the electricity demand for each region and showing an increase between the years
    \item The colors for the regions are the same as in the first chart, allowing the reader to easily connect the two charts
    \item The colors are again muted, where the color for 2020 is slightly darker to show that it is the more recent data
    \item Because I didn't want to label the two years for every region I used a small inset chart which shows that the lighter and smaller bars represent the data for 2000 and the darker and bigger bars represent the data for 2020
    \item The smallest and biggest increases are annotated to show that the increase happened everywhere and that Asia is the fastest growing region
\end{itemize}

\subsection{Chart 3: Change of electricity mix}

\begin{itemize}
    \item Shows electricity mixes for 2000 and 2020 using a treemap
    \item For showing the proportions of each element compared to the whole I first tried to use a stacked bar chart but it was not easy to read and understand
    \item Then I tried to use a pie chart but because we (humans) are not good at comparing angles the years were not easily comparable
    \item The treemap allowed me to show the proportions more easily and also allows for easy comparison between the years
    \item To highlight the renewable energy sources I chose a palette of different shades gray for the fossil fuels and distinct colors for the renewable energy sources:
    \begin{itemize}
        \item Hydro: Dark Blue. Because of the deep waters of oceans and lakes
        \item Wind: Slate Blue. Because of the earthy tone of the color which reminds one of the wind
        \item Solar: Yellow. Because of the sun
        \item Biofuel: Green. Because of the green forests and fields which are used to grow the materials of the biofuel
    \end{itemize}
    \item The two values smaller than 1\% are labeled with an arrow because the text would be too big to fit in the rectangle
    \item To better highlight the increase of renewable energy sources I used outlined the rectangles for the renewable energy sources with a faded yellow
\end{itemize}

\subsection{Chart 4: Change of cost vs. capacity for renewable energy sources}

\begin{itemize}
    \item Shows the decreasing costs and increasing installed capacities for solar, onshore and offshore wind energy from 2010 to 2022 using a connected scatter plot
    \item I chose a connected scatter plot because it allows me to show the relationship between the two variables and the evolution over time
    \item The colors for the renewable energy sources are the same as in the third chart, except that \textit{Wind} was split into \textit{Onshore Wind} and \textit{Offshore Wind}
    \item I additionally muted the colors of the lines connecting the dots and made a thicker connection from the first to the last dot to emphasize the evolution
    \item The logarithmic scales for the x and y axis are needed because the range of the values is really big
    \item The first and last dot are annotated with the price and the year to enable the reader to easily compare the values
    \item The labelling is done directly on the chart with texts that are in the same color as the corresponding line
    \item For each technology the learning rate (rate at which the cost decreases with each doubling of capacity) is annotated
\end{itemize}

\newpage

\section{Tooling}

The following section describes the tools and libraries used for this project, as well as the reasons for choosing them.

\subsection{Pandas}
Project URL: \texttt{https://pandas.pydata.org/}.\newline
Pandas is a powerful library for data analysis, cleaning and manipulation.
I used Pandas for data exploration and data cleaning.
It also provides a lot of useful functions for data visualization, but I did not use them for this project.

\subsection{NumPy}
Project URL: \texttt{https://numpy.org/}.\newline
NumPy is a library for scientific computing and provides a lot of useful functions for linear algebra, Fourier transform and random number capabilities.
I mainly used NumPy for the fast and efficient array operations.

\subsection{Matplotlib}
Project URL: \texttt{https://matplotlib.org/}.\newline
Matplotlib is a library for data visualization and plotting.
I used Matplotlib for creating the charts for this project.
I chose it over other libraries like Seaborn or Plotly because I already had experience with it and it is very flexible.
Furthermore, I like the fact that it is very low-level and gives you a lot of control over the charts.
Seaborn is already too opinionated for my taste and Plotly is too high-level.

\subsection{Squarify}
Project URL: \texttt{https://github.com/laserson/squarify}.\newline
Squarify is a pure Python implementation of the squarify treemap layout algorithm.
I used it for the third chart which shows the change of electricity mix between two years.
Again, I liked its simple to use API and the fact that is directly utilizes Matplotlib which allowed me to easily style the chart.
The Plotly-alternative Treemap is too high-level and does not allow for the same level of customization.

\subsection{Quarto}
Project URL: \texttt{https://quarto.org/}.\newline
Quarto is a document system that allows you to combine code, text and output in a single document.
I used it for this project to create the actual article containing the data story.
As my story is static and does not require any interactivity, I chose it over Streamlit or Dash.

\newpage

\section{Acknowledgements}

During my research I found many interesting resources that helped me understand the topic better and find the right datasets.
The following list contains the most important ones:

\begin{itemize}
    \item \textbf{Our World in Data} \newline
    \texttt{https://ourworldindata.org/} \newline
    Our World in Data is a scientific online publication that focuses on large global problems such as poverty, disease, hunger, climate change, war, existential risks, and inequality.
    It is run by a non-profit organization based in Oxford, England.
    I used it to find the datasets for life expectancy, infant mortality rate, polio cases, internet users and literacy rate.\newline
    Furthermore, its phenomenal \href{https://ourworldindata.org/cheap-renewables-growth}{article about the decreasing cost of renewable energy sources} inspired me to use it as the topic for my data story.

    \item \textbf{World Bank} \newline
    \texttt{https://data.worldbank.org/} \newline
    The World Bank is an international financial institution that provides loans and grants to the governments of low- and middle-income countries for the purpose of pursuing capital projects.
    I used it to find the datasets for life expectancy and infant mortality rate.

    \item \textbf{Kaggle} \newline
    \texttt{https://www.kaggle.com/} \newline
    Kaggle is an online community of data scientists and machine learners, owned by Google LLC.
    I used it to find the datasets for global internet users and global energy consumption.

    \item \textbf{Tidy Tuesday} \newline
    \texttt{https://www.tidytuesday.com/} \newline
    Tidy Tuesday is a weekly data project aimed at the R ecosystem.
    I used it as an inspiration for finding a topic for my story that is supported by high-quality datasets.
    The \href{https://github.com/rfordatascience/tidytuesday/blob/master/data/2023/2023-06-06/readme.md}{2023-06-06 edition} made me aware of the OWID dataset which I used for my story.

    \item \textbf{World Economic Forum} \newline
    \texttt{https://www.weforum.org/} \newline
    The World Economic Forum is an international NGO that engages with political, business, cultural and other leaders of society to shape global, regional and industry agendas.
    In February 2023 the published the article \textit{\href{https://www.weforum.org/agenda/2023/02/renewables-world-top-electricity-source-data/}{Renewables will be world's top electricity source within three years, IEA data reveals
    }} which inspired me to explore and compare the energy consumption and demand for world regions.
    
    \newpage
    \item \textbf{melodysheep} \newline
    \texttt{https://www.melodysheep.com/} \newline
    melodysheep is a YouTube channel run by John D. Boswell, a filmmaker, composer, VFX artist, and editor from the pacific northwest.
    I used it as inspiration for my data story.
\end{itemize}

\end{document}