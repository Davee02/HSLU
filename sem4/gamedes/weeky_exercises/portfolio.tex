\documentclass{article}

\usepackage[raggedrightboxes]{ragged2e}
\usepackage{hyperref}
\usepackage{graphicx}
\usepackage{xurl}
\usepackage[left=2cm,right=2cm,bottom=1.7cm]{geometry}
\graphicspath{ {./img/} }

\hypersetup{
    colorlinks=true,
    linkcolor=black
}

\title{%
Portfolio GAMEDES FS24\\
\large Zusammenfassende Abgabe der \\
  wöchenlichen Aufgaben}
\author{David Hodel}
\date{20.05.2024}

\begin{document}

\maketitle
\newpage

\tableofcontents
\newpage

\section{SW1: Game vs Toy, Patterns and Tropes, Spassfaktor}

\textbf{Ausgewähltes Spiel:} Portal 2
\textbf{Beschreibung:} "Portal 2" ist ein Puzzle-Plattformer-Videospiel, entwickelt und veröffentlicht von Valve.
Es ist der Nachfolger von "Portal" und wurde im Jahr 2011 veröffentlicht. Als Spieler übernimmt man die Rolle der Protagonistin
Chell in einem dystopischen Forschungslaboratorium, dem Aperture Science Enrichment Center. Das Kernspielprinzip dreht sich um die
Erstellung von Portalen, um physikbasierte Rätsel zu lösen. Diese Portale verbinden zwei entfernte Flächen und ermöglichen
unkonventionelle Bewegungen und Lösungswege durch die Spielumgebung.
\\
\includegraphics[width=11cm]{01_portal2_lasers.jpg}
\\
\includegraphics[width=11cm]{01_portal2_gel.jpg}
\\
\textbf{Klassifizierung:}
\begin{itemize}
    \item "Portal 2" ist klar als ein Spiel (Game) zu klassifizieren.
    \item Es hat feste Regeln und Ziele: Spieler müssen Rätsel lösen, um von einem Level zum nächsten zu gelangen.
    \item Das Spiel bietet eine strukturierte Erfahrung mit einer definierten Handlung und einem vorgegebenen Ende.
    \item Es gibt wenig Raum für freies, unstrukturiertes Spielen, was typisch für ein Spielzeug (Toy) wäre.
\end{itemize}
\bigskip
\textbf{Aussergewöhnliche Patterns/Tropes:}
\begin{itemize}
    \item \textbf{Loads of Loads of Loading:} Portal 2 hat viele Loading Screens, vor allem im ersten Teil des Spiels.
    Es wird zwar versucht, es mit dem Fahren eines Aufzuges zu verstecken, was aber nicht wirklich gelingt, da trotzdem immer
    "Loading" angezeigt wird und das Spiel für einen Moment anhält.
    \item \textbf{First-Person Ghost:} Chell hat weder einen Schatten noch kann man ihre Beine sehen, wenn man hinunterschaut.
    Schaut man jedoch durch ein Portal, ist ihr ganzer Körper zu sehen.
    \item \textbf{Magnet Hands:} Chell lässt die Portal-Gun nie los, egal ob sie gerade kopfüber von der Decke fällt oder andere
    waghalsige Stunts macht.
\end{itemize}
\bigskip
\textbf{Spassfaktor:} Mir gefällt das Lösen von anspruchsvollen Puzzles sehr. Durch das extensive Playtesting, für welches Valve
bekannt ist, haben die Levels genau den richtigen Schwierigkeitsgrad. Die Lernkurve ist sehr geschickt gewählt.
Auch das umgebungsbasierte Storytelling und die lustigen (und teilweise nervigen) Charaktere sind interessant.

\section{SW2: Parts, Rules und Loope}

\textbf{Spiel:} Portal 2
\\
\textbf{Wichtigste Spielelemente:}
\begin{itemize}
    \item \textbf{Portal Gun:} Kernelement des Spieles. Wird verwendet, um Portale zu erstellen, welche zum Lösen der Puzzles
    verwendet werden.
    \item \textbf{Testchambers:} Das Spiel besteht aus einer Reihe von Rätseln, die der Spieler lösen muss.
    \item \textbf{Cubes:} Der Spieler kann ca. 1x1 Meter grosse Würfel aufheben und innerhalb einer Chamber herumtragen, um damit
    Knöpfe auf dem Boden zu aktivieren, welche das Lösen des Puzzles ermöglichen.
    \item \textbf{Knöpfe:} Das Drücken von kleinen und grossen Knöpfen löst verschiedene Aktionen in den Chambers aus.
    \item \textbf{Bewegende Plattformen:} Sich vertikal oder horizontal bewegende Plattformen
    (ausgelöst durch den Spieler oder standardmässig) erlauben das Erreichen von weiteren Orten im Level.
\end{itemize}
\bigskip
\textbf{Secrets:}
\begin{itemize}
    \item Da Portal 2 ein Einzelspieler-Spiel ist, gibt es keine Geheimnisse zwischen Spielern, da es nur einen Spieler gibt.
    \item Das Spiel verheimlicht dem Spieler die Abfolge von Aktionen, welche zum Lösen einer Testchamber benötigt werden.
    \item Viele Rätsel lassen sich auf mehrere Wege lösen. Spieler müssen ihre eigene Strategie entwickeln, um Levels zu lösen.
    Diese Strategie wird vom Spiel nicht vorgegeben oder erzwungen.
\end{itemize}
\bigskip
\textbf{Meta-Loops}
\begin{itemize}
    \item Das vorgegebene Spielziel ist das Durchlaufen aller Testchambers.
    \item Der Spieler möchte ausserdem an die Freiheit / Oberfläche gelangen.
\end{itemize}
\bigskip
\textbf{Macro-Loops}
\begin{itemize}
    \item Die einzelnen Testchambers sind die Levels im Spiel und müssen nach Eintreten wieder verlassen werden.
    \item Bewegen zwischen Testchambers (manchmal nur kurzer Aufzug, manchmal längere Strecke durch Aperture-Labor mit narrativen
    Elementen).
\end{itemize}
\bigskip
\textbf{Micro-Loops}
\begin{itemize}
    \item Analyse der Chamber durch Beobachten und Informationen sammeln.
    \item Interaktion mit der Umwelt: Platzieren von Portals, Drücken von Knöpfen, Aufheben und Platzieren von Cubes.
    \item Bewegen durch die Testchambers.
\end{itemize}
\bigskip
\textbf{3 wichtigste Spielregeln}
\begin{itemize}
    \item Das Ziel ist das Durchlaufen der Testchambers. Jede Testchamber lässt sich durch das Verlassen durch eine markierte Türe
    abschliessen. Der Weg zum Ausgang ist jedoch nie direkt machbar. Zum Erreichen der Türe muss die Umgebung der Testchamber
    verändert werden oder der Spieler muss durch andere Aktionen den Ausgang erreichen.
    \item Die Spieler können mit der Portal-Gun zwei unterschiedliche Portale erzeugen, ein blaues und ein oranges. Diese Portale
    sind miteinander verbunden, sodass alles, was in das eine Portal hineingeht, durch das andere herauskommt. Dies beinhaltet den
    Spieler, Objekte und einige Formen von Energie wie Laserstrahlen. Die Oberflächen, auf denen Portale platziert werden können,
    sind oft begrenzt, was die Puzzellösung beeinflusst.
    \item Bewegung durch Portale behält den Impuls bei (oft im Spiel als "Speedy Thing goes in, Speedy Thing comes out" bezeichnet).
    Dies bedeutet, dass der Spieler und Objekte beim Durchtritt durch Portale ihre Geschwindigkeit und Richtung in Bezug auf das
    Portal beibehalten, was für das Lösen von Rätseln durch "Flings" oder andere momentum-basierte Tricks wichtig ist.
\end{itemize}

\includegraphics[width=0.9\textwidth]{02_portal2_gameloop.png}

\section{SW3: Spass und Player Types}

\textbf{4 Keys 2 Fun:} Die Kategorisierung von Spass in vier Hauptbereiche bietet meiner Meinung nach eine intuitive und klar
strukturierte Methode, um unterschiedliche Spass-Arten zu analysieren und auf eine Vielzahl von Erfahrungen anzuwenden.
Diese Einteilung hilft, ein Verständnis dafür zu entwickeln, wie Spieler auf diverse Spielmechaniken reagieren, indem
sie komplexe Erlebnisse vereinfacht darstellt. Während diese Kategorien die Konzepte einfach greifbar machen und eine direkte
Anwendung auf Spiele ermöglichen, weisen sie auch Grenzen auf. Ich finde, die vereinfachte Sichtweise kann zu einer
Übersimplifizierung führen, die möglicherweise nicht die gesamte Komplexität / Vielfalt des Spielerlebnisses einfängt.
Es besteht das Risiko, dass individuelle Vorlieben und nuancierte Aspekte des Spielspasses nicht vollständig erfasst werden.
\\
\\
\textbf{Gamer Motivation Model:} Dieses Modell fängt die vielen verschiedenen Typen für die Motivation, ein Spiel zu spielen,
meiner Meinung nach besser ein als andere Modelle. Im Vergleich zu "4 Keys 2 Fun" zeigt es nicht nur, was ein Spieler während dem
Spielen fühlt (welche Art von "Spass"), sondern auch, was ihn überhaupt dazu motiviert, das Spiel anzufassen. Ich finde es wichtig,
Menschen nicht in Schubladen wie "Herkunft", "Geschlecht", etc. zu stecken. Das Modell setzt dies prima um, indem es diese Faktoren
gänzlich ignoriert und nur auf die individuellen Präferenzen eingeht. Jedoch besteht auch hier die Gefahr der Übersimplifizierung,
da sich auch mit feiner granulierteren Kategorien und Motivationen nicht alle Facetten eines Spiels erklären lassen. Ich denke aber,
dass dies ein allgemeines Problem von jeglichen Modellen ist, da diese immer eine Simplifikation vom echten Leben darstellen.
\\
\\
\textbf{4 Keys 2 Fun anhand Portal 2:}
\begin{itemize}
    \item \textbf{Hard Fun}
    \begin{itemize}
        \item Die Rätsel, die die Spieler lösen müssen, indem sie die Portal-Gun verwenden, um von einem Ort zum anderen zu gelangen,
         Objekte zu bewegen oder Fallen zu vermeiden. Die Befriedigung kommt aus dem Gefühl der Leistung, wenn schwierige Rätsel
         gelöst werden.
    \end{itemize}
    \item \textbf{Easy Fun}
    \begin{itemize}
        \item Portal 2 bietet dies durch seine faszinierende und oft humorvolle Geschichte, die durch die Erkundung der Aperture
        Science Einrichtung und das Interagieren mit Charakteren wie GLaDOS und Wheatley erlebt wird. Die Umgebung selbst regt zur
        Neugier an und ermutigt Spieler, mit den Portalen zu experimentieren, um zu sehen, was passiert.
    \end{itemize}
    \item \textbf{Serious Fun}
    \begin{itemize}
        \item Spieler können durch Lösen der Rätsel logisches Denken und Problemlösungsfähigkeiten üben und verbessern.
        Es fördert kognitive Fähigkeiten wie räumliches Vorstellungsvermögen und Planung.
    \end{itemize}
    \item \textbf{People Fun}
    \begin{itemize}
        \item Im Koop-Modus müssen zwei Spieler zusammenarbeiten, um Rätsel zu lösen, die speziell dafür entworfen wurden, dass sie
        die Koordination und Kommunikation zwischen den Spielern erfordern.
        \item Es gibt eine rege Speedrunning-Community für das Spiel, welche zu gegenseitigen Wettkämpfen animiert.
    \end{itemize}
\end{itemize}

\section{SW4: MDA-Analyse}

\textbf{Gruppe:} Maiko Trede, Nevin Helfenstein, David Hodel
\\
\textbf{Arbeitsraum auf Miro:} \url{https://miro.com/app/board/uXjVNgK4rZY=/}
\\
\textbf{Spiele:} \textit{The Witcher 3} vs. \textit{Breath of the Wild}
\\
\includegraphics[width=1.0\textwidth]{04_mda.png}

\section{SW5: Game Design Dokument}

Die Analyse der Design-Dokumente für "Wasteland 2," "Rogue Legacy," und "The Nightmares of Edith Finch" zeigt die Vielfalt und
Individualität in der Spieleentwicklung. "Wasteland 2" betont den Aufbau einer post-apokalyptischen Welt mit detaillierten
Beschreibungen und Entscheidungsfreiheit, die zu einem komplexen Netz von Ursachen und Wirkungen führt. "Rogue Legacy" fokussiert
sich auf Charakterentwicklung und -interaktion, wobei die Individualität der NPCs und die Anpassbarkeit der Spielfiguren und der
Spielwelt hervorgehoben werden. "The Nightmares of Edith Finch" legt großen Wert auf die narrative Struktur und die nichtlineare
Erzählweise, die durch Erkundung und die Verwendung von Texten eine immersive und emotionale Erfahrung schafft. Diese Unterschiede
in den Design-Dokumenten reflektieren die verschiedenen Phasen und Zielsetzungen der Spiele und zeigen die Bedeutung gut
durchdachter Design-Dokumente, die als Grundlage für die kreative Vision dienen. Sie verdeutlichen, dass es keinen universellen
Ansatz für die Spielentwicklung gibt, sondern dass jedes Dokument den spezifischen Bedürfnissen und Zielen des Projekts angepasst
wird.

\section{SW6: Paper Prototype}

\textbf{Gruppe:} Maiko Trede, Nevin Helfenstein, David Hodel
\\
\textbf{Spiel:} Donkey Kong (NES)
\\
\textbf{Video:} \url{https://photos.app.goo.gl/wMDBEoE8J59iffc28}
\\ \\
Unser Papier Prototyp ist nahe am originalen Donkey Kong Spiel, welches als typisches "Arcade" / NES Game gilt.
Wir haben uns von dem originalen Spiel, welches \url{https://www.retrogames.cz/play_001-NES.php}{hier} gespielt werden kann,
inspirieren lassen.
\\
\textbf{Grundlegende Mechaniken, Dynamiken und Ästhetiken:}
\\
\textbf{Mechaniken:}
\begin{itemize}
    \item Leitermechaniken: Die Fähigkeit des Spielers, Leitern hoch- und runter zu klettern, um verschiedene Plattformen zuerreichen.
    \item Fass- und Hindernisbewältigung: Wie sich Fässer und andere Hindernisse bewegen und wie der Spieler diese überwinden
    oder ihnen ausweichen kann.
    \item Rettungsmechanik: Die Aktionen und Strategien, die erforderlich sind, um Pauline zu erreichen und zu retten.
    \item Punktesystem und Belohnungen: Die Art und Weise, wie Punkte gesammelt werden, durch das Aufsammeln von Gegenständen,
    das Überwinden von Hindernissen oder das Erreichen neuer Levels.
\end{itemize}
\bigskip
\textbf{Dynamiken:}
\begin{itemize}
    \item Feinddynamiken: Die Bewegungsmuster und Verhaltensweisen der Fässer und anderer Hindernisse, die Donkey Kong wirft.
    \item Spielerstrategie und Risikobewertung: Wie der Spieler Entscheidungen trifft basierend auf der Position und dem Verhalten
    der Hindernisse sowie dem Layout der Plattformen.
    \item Adaptionsdynamik: Wie sich das Spielverhalten ändert, wenn der Spieler fortschreitet, einschließlich der Geschwindigkeit
    und der Komplexität der Hindernisse.
\end{itemize}
\bigskip
\textbf{Ästhetik:}
\begin{itemize}
    \item Grafische Darstellung und Stil: Die Visualisierung von Charakteren, Plattformen, Hindernissen und der Umgebung, um eine
    einladende und herausfordernde Spielwelt zu schaffen.
    \item Musik und Soundeffekte: Die Rolle der Audioelemente in der Schaffung einer spannenden Atmosphäre und der Unterstützung
    der Spielmechaniken.
    \item Narrative und Charakterdesign: Die Darstellung der Beziehung zwischen den Charakteren (z.B. Mario, Pauline, Donkey Kong)
    und wie diese die Spielerfahrung und das Engagement beeinflusst.
\end{itemize}
\bigskip
\textbf{Testen des Prototypen:}
Wir haben das Video und den Papierprototypen mit einer anderen Gruppe besprochen und angeschaut. Die andere Gruppe meinte, dass das
Design aktuell noch sehr simpel und einfach sei. Weiter bemängelten sie die limitierten Implementationen. Also zum Beispiel,
dass es nur einen Typ von Fass gibt oder dass die lebendige Flamme, welche Mario jagt, nicht implementiert wurde. Jedoch meinte die
andere Gruppe, dass das Design sehr kreativ sei und dass David einen wunderschönen Donkey Kong gezeichnet hat.
Nächster Schritt wäre also die Implementierung verschiedener Fässer, die kleine Flamme, welche Mario jagt, und den Hammer als
Power-Up, sodass Mario die Fässer, welche ihm entgegenkommen, zerstören kann.

\section{SW7: Interesting Choices}

Bei "Mensch ärgere Dich nicht" startet Jeder Spieler mit vier Spielfiguren auf seinem Startfeld. Das Ziel des Spiels ist es, alle
eigenen Spielfiguren im Uhrzeigersinn um das Spielfeld zu bewegen und sicher in das Ziel zu bringen. Die Bewegungen der Spielfiguren
werden durch Würfeln bestimmt. Es muss eine Sechs gewürfelt werden muss, um eine Figur aus dem Startbereich auf das Spielfeld zu
bringen. Spieler können die Figuren der Gegner in den Startbereich zurückschlagen, indem sie auf das gleiche Feld ziehen.
\\
\includegraphics[width=8cm]{07_mensch_aergere_dich_nicht.jpg}
\\ \\
Dieses Spiel ist bekannt dafür, dass es stark vom Glück abhängt und oft keine tiefgreifenden strategischen Entscheidungen zu treffen
gibt. Die Hauptentscheidung im Spiel ist oft, welche Spielfigur bewegt wird. Diese Entscheidung wird jedoch meistens durch die
Würfe und die Positionen der anderen Spieler auf dem Brett bestimmt. Es handelt sich also um \textbf{offensichtliche
Entscheidungen} (häufig gibt es eine offensichtlich "richtige" Figur zu bewegen) und \textbf{keine 
Entscheidungen} (manchmal kann nur eine einzelne Figur bewegt werden, vor allem zu Beginn des Spieles).
\\ \\
Man könte das Spiel spannender gestalten, indem man das Bilden von temporären Allianzen zwischen Spielern erlauben würde.
Solche Allianzen können nützlich sein, um gemeinsam Feinde zu blockieren oder sich gegenseitig sicher durch "gefährliche" Bereiche
des Bretts zu navigieren.
Auch könnte man Ressourcenpunkte hinzufügen. Jeder Spieler startet das Spiel mit einer bestimmten Anzahl an Ressourcenpunkten. Diese
Punkte können verwendet werden, um spezielle Aktionen durchzuführen (zusätzlicher Würferwurf, Barriere errichten, Feind an Start zurückschicken, etc.).
Spieler können wählen, ob sie Ressourcenpunkte sparen möchten, um sie später für kritische Momente zu brauchen. Sie können die
Punkte jedoch auch zu Beginn schon ausgeben, um einen "Startboost" zu erhalten und so das Earlygame zu dominieren.

\section{SW8: Balancing}

\textbf{Spiel:} The Legend of Zelda: Breath of the Wild

\subsection{Waterblight Ganon}

\includegraphics[width=10cm]{08_waterblight_ganon.png}
\\
\textbf{Stärken:}
\begin{itemize}
    \item Kann große Entfernungen überbrücken (Speerangriffe)
    \item Besitzt die Fähigkeit, Eisblöcke zu beschwören, die dem Spieler Schaden zufügen und behindern
    \item Starke Angriffe im Wasser, wo er bessere Beweglichkeit und Reichweite hat
\end{itemize}
\bigskip
\textbf{Schwächen:}
\begin{itemize}
    \item Anfällig für elektrische Angriffe, besonders effektiv, wenn er im Wasser ist
    \item Langsam bei Nahkampfangriffen, was ermöglicht, auszuweichen und zu kontern
    \item Eisblöcke können mit der Cryonis-Rune zerstört werden, um den Angriff abzuwehren oder sogar umzukehren
\end{itemize}
\bigskip
\textbf{Balancing:}
\begin{itemize}
    \item Wasserumgebung verstärkt seine Angriffe, stellt jedoch auch eine Schwäche dar (wenn der Spieler elektrische Pfeile oder Waffen nutzt)
    \item Ermöglicht verschiedene Methoden zum Kämpfen: Fernkampf, Ausweichen und direkten Nahkampf
\end{itemize}

\subsection{Thunderblight  Ganon}

\includegraphics[width=10cm]{08_thunderblight_ganon.png}
\\
\textbf{Stärken:}
\begin{itemize}
    \item Extrem schnell und aggressiv, schwer vorhersehbare Bewegungen
    \item Kann metallische Stacheln beschwören, die elektrische Schläge auslösen
    \item Kann sich teleportieren, was ihn schwer zu treffen macht
\end{itemize}
\bigskip
\textbf{Schwächen:}
\begin{itemize}
    \item Anfällig für Angriffe während seiner Aufladephase, wenn er seine elektrische Energie sammelt
    \item Sein Schild kann durchschlagen werden, wenn man das richtige Timing beim Angreifen hat
    \item Schwächer im direkten Nahkampf, sobald sein Schild beschädigt ist
\end{itemize}
\bigskip
\textbf{Balancing:}
\begin{itemize}
    \item Schnelligkeit und Teleportation fordern präzises Timing und strategische Platzierung
    \item Elektrische Angriffe können vermieden oder abgewehrt werden
    \item Erfordert eine gute Beherrschung des Kampfsystems
\end{itemize}

\section{SW10: Narration}

\textbf{Spiel:} The Witcher 3: Wild Hunt
\\
Die Narration von The Witcher 3 ist eine Kombination von Branching und Threading:
\begin{enumerate}
  \item \textbf{Branching}: Dies ist die dominante Struktur in \textit{The Witcher 3}, da die Spieler zahlreiche Entscheidungen
  treffen, die zu deutlich unterschiedlichen Erzählwegen und Enden führen. Die Geschichte verzweigt sich an vielen Stellen,
  basierend auf den Entscheidungen, die der Spieler trifft, wie etwa welche Charaktere unterstützt werden, wie Quests gelöst werden
  und wie mit Schlüsselcharakteren interagiert wird.
  \item \textbf{Threading}: Zusätzlich zu den vielen Verzweigungen hat das Spiel auch eine verflochtene Erzählung, bei der mehrere
  Handlungsstränge parallel laufen und sich zu bestimmten Punkten kreuzen oder beeinflussen. Die Nebenquests und Hauptquests sind
  miteinander verknüpft und haben Auswirkungen aufeinander.
\end{enumerate}
\bigskip
Das Diagramm zeigt einige Schlüsselentscheidungen und Handlungsstränge, die Geralt auf seiner Suche nach seiner Ziehtochter Ciri
folgt. Die Story verzweigt sich an vielen Punkten, abhängig von Geralts Entscheidungen bezüglich Verbündeter, moralischer Dilemma
und Questergebnissen. Die Wahl der Entscheidungen führt zu verschiedenen Enden.
\\
\includegraphics[width=1.0\textwidth]{10_witcher3_narration.png}

\section{SW11: Design Patterns}

\subsection{Aufgabe 1: Kleines Jumpcrafter-Level}

\href{https://hsluzern-my.sharepoint.com/:u:/g/personal/david_hodel_stud_hslu_ch/Ea4OqG_pyeNEls8wVq7Zn8cB2uJQy9jvlJaBs3AFmZM3cw?e=mSX9NU}{Downloadlink}
\\ \\
\includegraphics[width=1.0\textwidth]{11_jumpcrafter_1.png}

\subsection{Aufgabe 3: Kishōtenketsu}

\textbf{Mechanik:} Doppelsprünge erlauben anderes Navigieren
\\
\href{https://hsluzern-my.sharepoint.com/:u:/g/personal/david_hodel_stud_hslu_ch/EWiT-lbI3OlHmm16AtbiQrsBpFmHQv5jdMYLqAvznqtalw?e=Had2Mw}{Downloadlink}
\\ \\
\textbf{Ki:} Einfaches Einführen von Doppelsprüngen mit nur kleiner Konsequenz (kein Game-Over).
\\
\includegraphics[width=10cm]{11_ki.png}
\\ \\
\textbf{Sho:} Es kommen nun schwierigere Sprünge und der Spieler kann durch die Spikes sterben. Sprünge müssen präziser sein als vorher.
\\
\includegraphics[width=10cm]{11_sho.png}
\\ \\
\textbf{Ten:} Der Boden bewegt sich nun auch. Der Spieler muss die bewegenden Plattformen navigieren können. Der Tod ist nun auch
nicht mehr durch Spikes, sondern durch das Fallen in das Void.
\\
\includegraphics[width=10cm]{11_ten.png}
\\ \\
\textbf{Ketsu:} Gemütlicher Levelabschluss mit sicherem Boden. Spieler kann Münzen mit Doppelsprüngen einsammeln.
\\
\includegraphics[width=10cm]{11_ketsu.png}
\\ \\
\textbf{2D Pattern:} Safe Zones, Gap Pattern

\end{document}
