\documentclass{article}

\usepackage[raggedrightboxes]{ragged2e}
\usepackage{hyperref}
\usepackage{graphicx}
\usepackage{xurl}
\usepackage[left=2cm,right=2cm,bottom=1.7cm]{geometry}
\graphicspath{ {./img/} }

\hypersetup{
    colorlinks=true,
    linkcolor=black
}

\title{%
Invertigo \\
\large Game Proposal für GAMEDES \\
  Abschlussprojekt FS24}
\author{David Hodel}
\date{14.05.2024}

\begin{document}

\maketitle
\newpage

\section{Spielbeschreibung}
Invertigo ist ein 2D-Puzzle-Plattformspiel, bei dem der Spieler die Schwerkraft manipulieren muss,
um Rätsel zu lösen und durch die Levels zu navigieren.
Die Spielerperspektive ist seitlich (Side-View), ähnlich wie bei klassischen Plattformspielen.
Der Spieler steuert einen Charakter, der mithilfe von speziellen Blöcken die Richtung der Schwerkraft zu ändern kann.
sodass "oben" und "unten" vertauscht werden können.
Dies wird genutzt, um verschiedene Hindernisse zu überwinden, von einfachen Plattformen bis hin zu komplexen Mechanismen,
die nur unter bestimmten Schwerkraftbedingungen funktionieren.
\\
Der Titel "Invertigo" spiegelt die desorientierende und verwirrende Erfahrung wider, die die Spieler durchleben,
wenn sie die Schwerkraft umkehren und somit die gesamte Spielwelt auf den Kopf stellen.

\section{Fokus des Spiels}

\begin{itemize}
    \item \textbf{Spielmechanik}: Das zentrale Element des Spiels ist die Schwerkraftwechsel-Mechanik.
    Spieler müssen lernen, diese Mechanik strategisch zu nutzen, um durch zunehmend komplex gestaltete Räume zu navigieren.
    \item \textbf{Leveldesign}: Jedes Level führt neue Elemente ein, die zuerst isoliert vorgestellt und dann in Kombinationen verwendet werden.
    Die vorhandenen Elemente werden in immer komplexeren Rätseln kombiniert, um die Spieler herauszufordern.
    Für das Spiel sollen insgesamt 4-6 Levels entwickelt werden.
\end{itemize}

\section{Elemente}

Die Levels sollen aufeinander aufbauen und eine steigende Schwierigkeit bieten.
Es sollen verschiedene Elemente eingeführt werden, die das Lösen von neuartigen Problemen ermöglichen.
\\
Folgende Elemente soll das Spiel enthalten:
\begin{itemize}
    \item \textbf{Gravitationsschalter}: Schaltflächen, die die Schwerkraft umkehren, wenn der Spieler sie berührt und aktiviert.
    \item \textbf{Einseitige Plattformen}: Plattformen, die nur von einer bestimmten Gravitationsrichtung aus zugänglich sind.
    Spieler müssen die Gravitation umkehren, um diese Plattformen zu erreichen. Wird versucht, sie von der falschen Seite zu erreichen, fallen sie durch.
    \item \textbf{Mobile Kisten}: Kisten, die sich durch den Spieler bewegen lassen und als Hilfsmittel dienen, um sonst unerreichbare Bereiche zu erreichen.
    Die Kisten werden ebenfalls durch die wechselnde Schwerkraft beeinflusst und fallen durch die doppelseitigen Plattformen.
    \item \textbf{Void}: Unter- und oberhalb der Spielwelt befindet sich ein "Void", das eine allgegenwärtige Gefahr darstellt. Fällt der Spieler hinein, stirbt er und muss das Level neu starten.
    \item \textbf{Schwerkraftfelder}: Bereiche, in denen die Schwerkraft stärker oder schwächer ist, was die Bewegung des Spielers beeinflusst.
\end{itemize}

\section{Grafikstil und Atmosphäre}

Der Grafikstil von "Invertigo" ist minimalistisch und abstrakt, um die Schwerpunkt auf die Spielmechanik zu legen. Die Spielwelt besteht aus einfachen geometrischen Formen und klaren Linien.
Es werden ausschliesslich Sprites in einer 1-Bit-Farbpalette verwendet (schwarz \& weiss) um die visuelle Darstellung zu vereinfachen.
Der Fokus liegt auf der Klarheit und Lesbarkeit der Spielwelt und auf den Mechaniken des Spiels.
\\
Die Atmosphäre des Spiels ist düster und mysteriös, um die desorientierende Natur der Schwerkraftwechsel zu betonen.
Die Spieler sollen sich absichtlich unwohl fühlen, wenn sie die Schwerkraft umkehren und die Welt auf den Kopf stellen.

\section{Techniken}
Das Spiel wird in der Engine \hyperlink{https://godotengine.org/}{Godot} entwickelt. Die Verwendung von Godot ist kostenlos und es existieren viele Einsteiger-Tutorials für 2D-Spiele.
Die Entwicklung erfolgt in der Programmiersprache C\#, welche von Godot von Haus aus unterstützt wird.

\section{Vergleich mit anderen Spielen}

Ähnliche Spiele, wie "VVVVVV" und "Gravity Guy", nutzen Gravitationswechsel als zentrales Spielelement.
"VVVVVV" verwendet einen kontinuierlichen Wechsel zwischen Boden und Decke, während "Gravity Guy" schnelle Wechsel zwischen Laufen auf dem Boden und der Decke erlaubt.
\\
Im Vergleich zu diesen zwei spielen soll Invertigo jedoch eine stärkere Integration von Umgebungsrätseln, bei denen der Schwerkraftwechsel nicht nur zur Navigation,
sondern auch zur Lösung von Rätseln durch Interaktion mit Objekten und Mechanismen in der Spielwelt genutzt wird, haben.
Ausserdem erfolgt der Schwerkraftwechsel bei Invertigo nicht kontinuierlich, sondern durch Schalter, die vom Spieler aktiviert werden.
\\
Auch "Portal" und "Portal 2" nutzen Gravitation als zentrales Element, um Räume (welche abgekapselte Rätsel sind) auf nicht-lineare Weise zu überqueren.
Dort liegt jedoch der Fokus ganz klar auf den Portalen. Die Gravitation ist mehr ein Hilfsmittel, um die Portale effektiv zu nutzen.
Ausserdem sind die Portal-Spiele in 3D, was eine andere Art von Rätseln ermöglicht.

\section{Notwendige Entwicklungsschritte und Assets}
\subsection{Asset-Liste}
Es wird das kostenfreie \hyperlink{https://kenney.nl/assets/1-bit-platformer-pack}{1-Bit Platformer Pack} von Kenney für jegliche Sprites verwendet.
Dieses enthält insgesamt 400 Tiles in der Grösse 16x16 Pixel und kann direkt in Godot importiert werden.
\\
Die folgenden Assets werden aus der Tilemap benötigt
\begin{itemize}
    \item Spielcharakter
    \item Boden
    \item Wände
    \item Decke
    \item Gravitationsschalter
    \item Einseitige Plattformen
    \item Mobile Kisten
    \item Levelziel
\end{itemize}

Im folgenden Bild sind alle Assets aus dem 1-Bit Platformer Pack sichtbar:

\includegraphics[width=11cm]{kenney_assets}

Ein beispielhaftes Level könnte dann wie folgt aussehen:

\includegraphics[width=11cm]{example_level.png}

\subsection{Skripting}
Die Entwicklung von "Invertigo" erfordert die Erstellung von Skripten in C\#, um die Spielmechanik und das Leveldesign zu implementieren.
Die Skripte umfassen:
\begin{itemize}
    \item \textbf{Spielersteuerung}: Grundlegende Steuerung des Spielcharakters (laufen, springen) inkl. Kamera
    \item \textbf{Gravitationsschalter}: Aktivierung der Schwerkraftwechsel-Mechanik
    \item \textbf{Kistensteuerung}: Bewegung der Kisten durch den Spieler und Interaktion mit der Schwerkraft
    \item \textbf{Einseitige Plattformen}: Überprüfung, ob der Spieler von der richtigen Seite auf die Plattformen zugreift
    \item \textbf{Level-Manager}: Anzeigen des neuen Levels, wenn der Spieler das Ziel erreicht
    \item \textbf{Void-Check}: Überprüfung, ob der Spieler in den "Void" gefallen ist und das Level neu starten
    \item \textbf{Schwerkraftfelder}: Anpassung der Spielerbewegung basierend auf der Schwerkraftstärke
    \item \textbf{Kollisionsabfrage}: Überprüfung der Kollisionen zwischen dem Spieler und den Objekten in der Spielwelt
\end{itemize}

\end{document}